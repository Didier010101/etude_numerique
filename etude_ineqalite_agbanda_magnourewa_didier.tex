\documentclass[11pt]{article}
\usepackage[utf8]{inputenc}
\usepackage[T1] {fontenc}
\usepackage{amsmath,amssymb,amsthm}
\usepackage{geometry}
\geometry{margin =2.5cm}
\usepackage[french]{babel}
\usepackage{listings}
\usepackage{fancyhdr}
\usepackage{titlesec}
\usepackage{setspace}
\usepackage{mathtools}
\usepackage{bm}
\usepackage{courier}
\lstset{
language=Python,
basicstyle=\ttfamily\small,
frame=single,
breaklines=true,
showstringspaces=false}
\title{\textbf{\'ETUDE NUMÉRIQUE DE L'INÉGALITÉ $(i)^p + (i+1)^p<(i+2)^p$}}
\author {AGBANDA MAGNOUR\'EWA DIDIER,L3 Mathématiques,UNIV\'ERSIT\'E DE LOM\'E}
\date{\today}
\begin{document}

\maketitle

\begin{abstract}
Tout commence avec cet exercice: Comparer les deux nombres :$ 2^{100}+3^{100 }et 4^{100}$. 
Pour sa résolution nous pouvons procéder comme suit: $$2^{100} +3^{100 }< 3^{100}+3^{100 }= 2.3^{100}
= 2.3^3.3^{97} = 54.3^{97}
< 64.4^{97} = 4^3.4^{97} = 4^{100}$$
Alors la curiosité ici est de savoir ce qui se passait si nous essayons  de faire la m\^eme comparaison avec trois autres entiers naturels consécutifs.
Dans ce article, j' étudies numériquement et j'essaies de voir le comportement global de  l'inéquation \[(i)^p + (i+1)^p<(i+2)^p\] où bien-s\^ur $i$ et $p$ sont des entiers naturels. Ici mon étude ne portera pas uniquement sur la variations des trois éléments consécutifs mais mais aussi sur la variation de l'exposant.
\end{abstract}
\section{Introduction}
Les inégalités faisant intervenir les puissances sont assez fréquentes dans le monde des mathématiques ,bien que certaines de ces inégalités aient une forme simple et élémentaire , leur comportement peut dépendre de manière subtile de leurs paramètres. Je vais m'intéresser ici à l'inéquation \[(i)^p + (i+1)^p<(i+2)^p\] où $i$ et $p$ sont des entiers naturels. D'abord nous allons vérifier si cette inéquation est vraie pour tout $p$ \[\in \{1,2, \dots,100\},\]. Si c'est le cas alors nous affirmons que l'inégalité est donc vraie pour $p$ compris entre 0 et 100 ainsi nous pourrons essayer plus tard d'étendre cette inégalité sur un intervalle plus grand et pourquoi pas apporter une démonstration math\'ematique de sa véracité quelque soit $p$?. Si l'inégalité n'est pas vérifiée alors j'essayerai de trouver une relation nécessaire et suffisante entre i et p pour qu'elle reste vraie.
\section {Cadre de l'étude}
 Rappelons que dans toute la suite on considère :
\[i \in \{0,1,2, \dots,100\},\quad
p \in \{1,2, \dots,100\}.\]
(Je changerai l'intervalle quand besoin sera.)
Et que l'étude est menée selon 02 axes:
\begin{itemize}
 \item pour une valeur fixée  de $p$,on examine la validité de l'inégalité pour toutes les valeurs de $i$ comprises entre $0$ et $100$;on répète cette étude pour plusieurs valeurs entières de $p$;
 \item on cherche un intervalle de validité de l'inéquation
 \end{itemize}

\section {Observations élémentaires}
Remarquons que:
\subsection*{Cas $p= 1$}
Dans ce cas , l'inégalité devient \[i+(i+1)<i+2,\] ce qui équivaut à $2i+1<i+2$. Et bien sur cette inégalité est fausse pour tout $i \geq 1$(mais vraie pour i=0)
\subsection *{Cas $p=2$}
Pour $p=2$,on obtient \[i^2+(i+1)^2<(i+2)^2.\] ce qui est vérifiée juste pour un nombre fini de valeurs de $i$.(vraie pour i=0,1,2)

Ces premiers exemples montrent que l'inégalité ne peut être vraie pour toutes les valeurs de $i$ pour un $p$ fixé.
Je vais me tourner vers le deuxième volet de mon étude, trouver une relation entre $p$ et $i$ pour laquelle l'inégalité est vérifiée. pour cela voici les codes python nécessaires:
\section{Implémentation python}
Cette section présente le code python utilisé pour tester numériquement l'inégalité étudiée.
\begin{lstlisting}
def inegalite(i,p):
	return i**p+(i+1)**p<(i+2)**p
\end {lstlisting}
\subsection{\'Etude pour une valeur fix\'ee de $p$}
\begin {lstlisting}
def valeur_p_fixe(p,i_max=100):
	results=[]
	for i in range(0,101):
		if inegalite(i,p):
			results.append(i)
	return results
\end{lstlisting}
Cette fonction renvoie la liste des entiers $i$ pour lesquels l'inégalité est vérifiée pour une valeur de $p$.
\subsection {Étude pour toutes les valeurs de \texorpdfstring{$p$}{p}}
\begin {lstlisting}
def Etude_de_tout_p(p_max=100,i_max=100):
    test={}
    for p in range(1,p_max+1):
        for i in range(0,i_max+1):
		test[p-1]=len(valeur_p_fixe(p)
    return test
\end{lstlisting}
Cette fonction renvoie le nombre de $i$ compris entre 0 et 100 pour lesquels l'inégalité est vérifiée:
Par exemple nous pouvons remarquer que pour $p=6$ nombre de $i$=12; pour p=14, le nombre de $i$ =28
\section {Observations et Conjecture}
Nous constatons que l'inégalité reste vraie pour un certain  nombre fini de valeurs de $i$ qui n'est pas le même pour différentes valeurs de $p$. Ma remarque est donc celle ci: \bf{pour tout $i$ et $p$, entiers naturels, l'inégalité $(i)^p + (i+1)^p<(i+2)^p$ reste vraie pour   ``($2p-1$)'' $i$ consécutifs en partant de 0. L'on pourrait s'en convaincre en augmentant l'intervalle de variation de $i$ et de $p$.
\section{Conclusion}
Cette étude numérique sur l'inégalité $(i)^p + (i+1)^p<(i+2)^p$ nous montre que toutes les valeurs de $i$ n'obéissent pas à cette inégalité, mais nous rassure que les $2p-1$ premiers $i$ quant à eux ,vérifient l'inéquation. Ainsi je pourrais déclarer sans vérification que $$(20)^{20 }+ (21)^{20}<(22)^{20}$$ car i=20 fais partir des 39($2* p-1$ ) premiers termes , ou encore que $$(100)^{100 }+ (101)^{100}<(102)^{100}$$ et cela parce que $100 \in [0,2*p-1]$. 




\section{Bibliographie}
-OBJECTIF OLYMPIADES DE MATH\'EMATIQUE,TOME1:ALGEBRE \bf{Mohammed Aassila}

	
			










\end{document}
Delete etude_ineqalite_agbanda_magnourewa_didier.tex

